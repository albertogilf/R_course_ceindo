% Options for packages loaded elsewhere
\PassOptionsToPackage{unicode}{hyperref}
\PassOptionsToPackage{hyphens}{url}
%
\documentclass[
]{article}
\usepackage{lmodern}
\usepackage{amssymb,amsmath}
\usepackage{ifxetex,ifluatex}
\ifnum 0\ifxetex 1\fi\ifluatex 1\fi=0 % if pdftex
  \usepackage[T1]{fontenc}
  \usepackage[utf8]{inputenc}
  \usepackage{textcomp} % provide euro and other symbols
\else % if luatex or xetex
  \usepackage{unicode-math}
  \defaultfontfeatures{Scale=MatchLowercase}
  \defaultfontfeatures[\rmfamily]{Ligatures=TeX,Scale=1}
\fi
% Use upquote if available, for straight quotes in verbatim environments
\IfFileExists{upquote.sty}{\usepackage{upquote}}{}
\IfFileExists{microtype.sty}{% use microtype if available
  \usepackage[]{microtype}
  \UseMicrotypeSet[protrusion]{basicmath} % disable protrusion for tt fonts
}{}
\makeatletter
\@ifundefined{KOMAClassName}{% if non-KOMA class
  \IfFileExists{parskip.sty}{%
    \usepackage{parskip}
  }{% else
    \setlength{\parindent}{0pt}
    \setlength{\parskip}{6pt plus 2pt minus 1pt}}
}{% if KOMA class
  \KOMAoptions{parskip=half}}
\makeatother
\usepackage{xcolor}
\IfFileExists{xurl.sty}{\usepackage{xurl}}{} % add URL line breaks if available
\IfFileExists{bookmark.sty}{\usepackage{bookmark}}{\usepackage{hyperref}}
\hypersetup{
  pdftitle={Instalando y usando paquetes},
  hidelinks,
  pdfcreator={LaTeX via pandoc}}
\urlstyle{same} % disable monospaced font for URLs
\usepackage[margin=1in]{geometry}
\usepackage{color}
\usepackage{fancyvrb}
\newcommand{\VerbBar}{|}
\newcommand{\VERB}{\Verb[commandchars=\\\{\}]}
\DefineVerbatimEnvironment{Highlighting}{Verbatim}{commandchars=\\\{\}}
% Add ',fontsize=\small' for more characters per line
\usepackage{framed}
\definecolor{shadecolor}{RGB}{248,248,248}
\newenvironment{Shaded}{\begin{snugshade}}{\end{snugshade}}
\newcommand{\AlertTok}[1]{\textcolor[rgb]{0.94,0.16,0.16}{#1}}
\newcommand{\AnnotationTok}[1]{\textcolor[rgb]{0.56,0.35,0.01}{\textbf{\textit{#1}}}}
\newcommand{\AttributeTok}[1]{\textcolor[rgb]{0.77,0.63,0.00}{#1}}
\newcommand{\BaseNTok}[1]{\textcolor[rgb]{0.00,0.00,0.81}{#1}}
\newcommand{\BuiltInTok}[1]{#1}
\newcommand{\CharTok}[1]{\textcolor[rgb]{0.31,0.60,0.02}{#1}}
\newcommand{\CommentTok}[1]{\textcolor[rgb]{0.56,0.35,0.01}{\textit{#1}}}
\newcommand{\CommentVarTok}[1]{\textcolor[rgb]{0.56,0.35,0.01}{\textbf{\textit{#1}}}}
\newcommand{\ConstantTok}[1]{\textcolor[rgb]{0.00,0.00,0.00}{#1}}
\newcommand{\ControlFlowTok}[1]{\textcolor[rgb]{0.13,0.29,0.53}{\textbf{#1}}}
\newcommand{\DataTypeTok}[1]{\textcolor[rgb]{0.13,0.29,0.53}{#1}}
\newcommand{\DecValTok}[1]{\textcolor[rgb]{0.00,0.00,0.81}{#1}}
\newcommand{\DocumentationTok}[1]{\textcolor[rgb]{0.56,0.35,0.01}{\textbf{\textit{#1}}}}
\newcommand{\ErrorTok}[1]{\textcolor[rgb]{0.64,0.00,0.00}{\textbf{#1}}}
\newcommand{\ExtensionTok}[1]{#1}
\newcommand{\FloatTok}[1]{\textcolor[rgb]{0.00,0.00,0.81}{#1}}
\newcommand{\FunctionTok}[1]{\textcolor[rgb]{0.00,0.00,0.00}{#1}}
\newcommand{\ImportTok}[1]{#1}
\newcommand{\InformationTok}[1]{\textcolor[rgb]{0.56,0.35,0.01}{\textbf{\textit{#1}}}}
\newcommand{\KeywordTok}[1]{\textcolor[rgb]{0.13,0.29,0.53}{\textbf{#1}}}
\newcommand{\NormalTok}[1]{#1}
\newcommand{\OperatorTok}[1]{\textcolor[rgb]{0.81,0.36,0.00}{\textbf{#1}}}
\newcommand{\OtherTok}[1]{\textcolor[rgb]{0.56,0.35,0.01}{#1}}
\newcommand{\PreprocessorTok}[1]{\textcolor[rgb]{0.56,0.35,0.01}{\textit{#1}}}
\newcommand{\RegionMarkerTok}[1]{#1}
\newcommand{\SpecialCharTok}[1]{\textcolor[rgb]{0.00,0.00,0.00}{#1}}
\newcommand{\SpecialStringTok}[1]{\textcolor[rgb]{0.31,0.60,0.02}{#1}}
\newcommand{\StringTok}[1]{\textcolor[rgb]{0.31,0.60,0.02}{#1}}
\newcommand{\VariableTok}[1]{\textcolor[rgb]{0.00,0.00,0.00}{#1}}
\newcommand{\VerbatimStringTok}[1]{\textcolor[rgb]{0.31,0.60,0.02}{#1}}
\newcommand{\WarningTok}[1]{\textcolor[rgb]{0.56,0.35,0.01}{\textbf{\textit{#1}}}}
\usepackage{graphicx,grffile}
\makeatletter
\def\maxwidth{\ifdim\Gin@nat@width>\linewidth\linewidth\else\Gin@nat@width\fi}
\def\maxheight{\ifdim\Gin@nat@height>\textheight\textheight\else\Gin@nat@height\fi}
\makeatother
% Scale images if necessary, so that they will not overflow the page
% margins by default, and it is still possible to overwrite the defaults
% using explicit options in \includegraphics[width, height, ...]{}
\setkeys{Gin}{width=\maxwidth,height=\maxheight,keepaspectratio}
% Set default figure placement to htbp
\makeatletter
\def\fps@figure{htbp}
\makeatother
\setlength{\emergencystretch}{3em} % prevent overfull lines
\providecommand{\tightlist}{%
  \setlength{\itemsep}{0pt}\setlength{\parskip}{0pt}}
\setcounter{secnumdepth}{-\maxdimen} % remove section numbering

\title{Instalando y usando paquetes}
\author{}
\date{\vspace{-2.5em}}

\begin{document}
\maketitle

{
\setcounter{tocdepth}{2}
\tableofcontents
}
\hypertarget{los-paquetes}{%
\section{1. Los Paquetes}\label{los-paquetes}}

Los paquetes en R son colecciones de funciones y datos generados por la
comunidad. Algunos vienen por defecto con el entorno de R, mientras que
otros hay que instalarlos desde repositorios. Los principales
repositorios son:

\begin{itemize}
\tightlist
\item
  CRAN \url{https://cran.r-project.org/} el repositorio oficial de R.
\item
  Bioconductor \url{https://www.bioconductor.org/}, que aloja paquetes
  relacionados con la bioinformática.
\item
  Github \url{https://github.com/} no está relacionado con R
  específicamente, pero es muy popular para proyectos de código abierto.
\end{itemize}

Podemos ver los paquetes cargados en el entorno con el comando
\texttt{search()}:

\begin{Shaded}
\begin{Highlighting}[]
\KeywordTok{search}\NormalTok{()}
\end{Highlighting}
\end{Shaded}

\begin{verbatim}
## [1] ".GlobalEnv"        "package:stats"     "package:graphics" 
## [4] "package:grDevices" "package:utils"     "package:datasets" 
## [7] "package:methods"   "Autoloads"         "package:base"
\end{verbatim}

Los paquetes que vemos en este momento son los que están cargados por
defecto. En el paquete \texttt{datasets} hay conjuntos de datos de
ejemplo que se pueden cargar con el comando \texttt{data()}:

\begin{Shaded}
\begin{Highlighting}[]
\KeywordTok{data}\NormalTok{(iris)}
\NormalTok{iris}
\end{Highlighting}
\end{Shaded}

\begin{verbatim}
##     Sepal.Length Sepal.Width Petal.Length Petal.Width    Species
## 1            5.1         3.5          1.4         0.2     setosa
## 2            4.9         3.0          1.4         0.2     setosa
## 3            4.7         3.2          1.3         0.2     setosa
## 4            4.6         3.1          1.5         0.2     setosa
## 5            5.0         3.6          1.4         0.2     setosa
## 6            5.4         3.9          1.7         0.4     setosa
## 7            4.6         3.4          1.4         0.3     setosa
## 8            5.0         3.4          1.5         0.2     setosa
## 9            4.4         2.9          1.4         0.2     setosa
## 10           4.9         3.1          1.5         0.1     setosa
## 11           5.4         3.7          1.5         0.2     setosa
## 12           4.8         3.4          1.6         0.2     setosa
## 13           4.8         3.0          1.4         0.1     setosa
## 14           4.3         3.0          1.1         0.1     setosa
## 15           5.8         4.0          1.2         0.2     setosa
## 16           5.7         4.4          1.5         0.4     setosa
## 17           5.4         3.9          1.3         0.4     setosa
## 18           5.1         3.5          1.4         0.3     setosa
## 19           5.7         3.8          1.7         0.3     setosa
## 20           5.1         3.8          1.5         0.3     setosa
## 21           5.4         3.4          1.7         0.2     setosa
## 22           5.1         3.7          1.5         0.4     setosa
## 23           4.6         3.6          1.0         0.2     setosa
## 24           5.1         3.3          1.7         0.5     setosa
## 25           4.8         3.4          1.9         0.2     setosa
## 26           5.0         3.0          1.6         0.2     setosa
## 27           5.0         3.4          1.6         0.4     setosa
## 28           5.2         3.5          1.5         0.2     setosa
## 29           5.2         3.4          1.4         0.2     setosa
## 30           4.7         3.2          1.6         0.2     setosa
## 31           4.8         3.1          1.6         0.2     setosa
## 32           5.4         3.4          1.5         0.4     setosa
## 33           5.2         4.1          1.5         0.1     setosa
## 34           5.5         4.2          1.4         0.2     setosa
## 35           4.9         3.1          1.5         0.2     setosa
## 36           5.0         3.2          1.2         0.2     setosa
## 37           5.5         3.5          1.3         0.2     setosa
## 38           4.9         3.6          1.4         0.1     setosa
## 39           4.4         3.0          1.3         0.2     setosa
## 40           5.1         3.4          1.5         0.2     setosa
## 41           5.0         3.5          1.3         0.3     setosa
## 42           4.5         2.3          1.3         0.3     setosa
## 43           4.4         3.2          1.3         0.2     setosa
## 44           5.0         3.5          1.6         0.6     setosa
## 45           5.1         3.8          1.9         0.4     setosa
## 46           4.8         3.0          1.4         0.3     setosa
## 47           5.1         3.8          1.6         0.2     setosa
## 48           4.6         3.2          1.4         0.2     setosa
## 49           5.3         3.7          1.5         0.2     setosa
## 50           5.0         3.3          1.4         0.2     setosa
## 51           7.0         3.2          4.7         1.4 versicolor
## 52           6.4         3.2          4.5         1.5 versicolor
## 53           6.9         3.1          4.9         1.5 versicolor
## 54           5.5         2.3          4.0         1.3 versicolor
## 55           6.5         2.8          4.6         1.5 versicolor
## 56           5.7         2.8          4.5         1.3 versicolor
## 57           6.3         3.3          4.7         1.6 versicolor
## 58           4.9         2.4          3.3         1.0 versicolor
## 59           6.6         2.9          4.6         1.3 versicolor
## 60           5.2         2.7          3.9         1.4 versicolor
## 61           5.0         2.0          3.5         1.0 versicolor
## 62           5.9         3.0          4.2         1.5 versicolor
## 63           6.0         2.2          4.0         1.0 versicolor
## 64           6.1         2.9          4.7         1.4 versicolor
## 65           5.6         2.9          3.6         1.3 versicolor
## 66           6.7         3.1          4.4         1.4 versicolor
## 67           5.6         3.0          4.5         1.5 versicolor
## 68           5.8         2.7          4.1         1.0 versicolor
## 69           6.2         2.2          4.5         1.5 versicolor
## 70           5.6         2.5          3.9         1.1 versicolor
## 71           5.9         3.2          4.8         1.8 versicolor
## 72           6.1         2.8          4.0         1.3 versicolor
## 73           6.3         2.5          4.9         1.5 versicolor
## 74           6.1         2.8          4.7         1.2 versicolor
## 75           6.4         2.9          4.3         1.3 versicolor
## 76           6.6         3.0          4.4         1.4 versicolor
## 77           6.8         2.8          4.8         1.4 versicolor
## 78           6.7         3.0          5.0         1.7 versicolor
## 79           6.0         2.9          4.5         1.5 versicolor
## 80           5.7         2.6          3.5         1.0 versicolor
## 81           5.5         2.4          3.8         1.1 versicolor
## 82           5.5         2.4          3.7         1.0 versicolor
## 83           5.8         2.7          3.9         1.2 versicolor
## 84           6.0         2.7          5.1         1.6 versicolor
## 85           5.4         3.0          4.5         1.5 versicolor
## 86           6.0         3.4          4.5         1.6 versicolor
## 87           6.7         3.1          4.7         1.5 versicolor
## 88           6.3         2.3          4.4         1.3 versicolor
## 89           5.6         3.0          4.1         1.3 versicolor
## 90           5.5         2.5          4.0         1.3 versicolor
## 91           5.5         2.6          4.4         1.2 versicolor
## 92           6.1         3.0          4.6         1.4 versicolor
## 93           5.8         2.6          4.0         1.2 versicolor
## 94           5.0         2.3          3.3         1.0 versicolor
## 95           5.6         2.7          4.2         1.3 versicolor
## 96           5.7         3.0          4.2         1.2 versicolor
## 97           5.7         2.9          4.2         1.3 versicolor
## 98           6.2         2.9          4.3         1.3 versicolor
## 99           5.1         2.5          3.0         1.1 versicolor
## 100          5.7         2.8          4.1         1.3 versicolor
## 101          6.3         3.3          6.0         2.5  virginica
## 102          5.8         2.7          5.1         1.9  virginica
## 103          7.1         3.0          5.9         2.1  virginica
## 104          6.3         2.9          5.6         1.8  virginica
## 105          6.5         3.0          5.8         2.2  virginica
## 106          7.6         3.0          6.6         2.1  virginica
## 107          4.9         2.5          4.5         1.7  virginica
## 108          7.3         2.9          6.3         1.8  virginica
## 109          6.7         2.5          5.8         1.8  virginica
## 110          7.2         3.6          6.1         2.5  virginica
## 111          6.5         3.2          5.1         2.0  virginica
## 112          6.4         2.7          5.3         1.9  virginica
## 113          6.8         3.0          5.5         2.1  virginica
## 114          5.7         2.5          5.0         2.0  virginica
## 115          5.8         2.8          5.1         2.4  virginica
## 116          6.4         3.2          5.3         2.3  virginica
## 117          6.5         3.0          5.5         1.8  virginica
## 118          7.7         3.8          6.7         2.2  virginica
## 119          7.7         2.6          6.9         2.3  virginica
## 120          6.0         2.2          5.0         1.5  virginica
## 121          6.9         3.2          5.7         2.3  virginica
## 122          5.6         2.8          4.9         2.0  virginica
## 123          7.7         2.8          6.7         2.0  virginica
## 124          6.3         2.7          4.9         1.8  virginica
## 125          6.7         3.3          5.7         2.1  virginica
## 126          7.2         3.2          6.0         1.8  virginica
## 127          6.2         2.8          4.8         1.8  virginica
## 128          6.1         3.0          4.9         1.8  virginica
## 129          6.4         2.8          5.6         2.1  virginica
## 130          7.2         3.0          5.8         1.6  virginica
## 131          7.4         2.8          6.1         1.9  virginica
## 132          7.9         3.8          6.4         2.0  virginica
## 133          6.4         2.8          5.6         2.2  virginica
## 134          6.3         2.8          5.1         1.5  virginica
## 135          6.1         2.6          5.6         1.4  virginica
## 136          7.7         3.0          6.1         2.3  virginica
## 137          6.3         3.4          5.6         2.4  virginica
## 138          6.4         3.1          5.5         1.8  virginica
## 139          6.0         3.0          4.8         1.8  virginica
## 140          6.9         3.1          5.4         2.1  virginica
## 141          6.7         3.1          5.6         2.4  virginica
## 142          6.9         3.1          5.1         2.3  virginica
## 143          5.8         2.7          5.1         1.9  virginica
## 144          6.8         3.2          5.9         2.3  virginica
## 145          6.7         3.3          5.7         2.5  virginica
## 146          6.7         3.0          5.2         2.3  virginica
## 147          6.3         2.5          5.0         1.9  virginica
## 148          6.5         3.0          5.2         2.0  virginica
## 149          6.2         3.4          5.4         2.3  virginica
## 150          5.9         3.0          5.1         1.8  virginica
\end{verbatim}

Podemos obtener ayuda sobre cualquier comando con \texttt{?} (observa el
panel inferior derecho):

\begin{Shaded}
\begin{Highlighting}[]
\NormalTok{?data}
\end{Highlighting}
\end{Shaded}

\begin{verbatim}
## starting httpd help server ... done
\end{verbatim}

Con el operador \texttt{\$} podemos acceder a cada una de las variables
del conjunto de datos Iris:

\begin{Shaded}
\begin{Highlighting}[]
\NormalTok{iris}\OperatorTok{$}\NormalTok{Sepal.Width}
\end{Highlighting}
\end{Shaded}

\begin{verbatim}
##   [1] 3.5 3.0 3.2 3.1 3.6 3.9 3.4 3.4 2.9 3.1 3.7 3.4 3.0 3.0 4.0 4.4 3.9 3.5
##  [19] 3.8 3.8 3.4 3.7 3.6 3.3 3.4 3.0 3.4 3.5 3.4 3.2 3.1 3.4 4.1 4.2 3.1 3.2
##  [37] 3.5 3.6 3.0 3.4 3.5 2.3 3.2 3.5 3.8 3.0 3.8 3.2 3.7 3.3 3.2 3.2 3.1 2.3
##  [55] 2.8 2.8 3.3 2.4 2.9 2.7 2.0 3.0 2.2 2.9 2.9 3.1 3.0 2.7 2.2 2.5 3.2 2.8
##  [73] 2.5 2.8 2.9 3.0 2.8 3.0 2.9 2.6 2.4 2.4 2.7 2.7 3.0 3.4 3.1 2.3 3.0 2.5
##  [91] 2.6 3.0 2.6 2.3 2.7 3.0 2.9 2.9 2.5 2.8 3.3 2.7 3.0 2.9 3.0 3.0 2.5 2.9
## [109] 2.5 3.6 3.2 2.7 3.0 2.5 2.8 3.2 3.0 3.8 2.6 2.2 3.2 2.8 2.8 2.7 3.3 3.2
## [127] 2.8 3.0 2.8 3.0 2.8 3.8 2.8 2.8 2.6 3.0 3.4 3.1 3.0 3.1 3.1 3.1 2.7 3.2
## [145] 3.3 3.0 2.5 3.0 3.4 3.0
\end{verbatim}

\begin{Shaded}
\begin{Highlighting}[]
\NormalTok{iris}\OperatorTok{$}\NormalTok{Sepal.Length}
\end{Highlighting}
\end{Shaded}

\begin{verbatim}
##   [1] 5.1 4.9 4.7 4.6 5.0 5.4 4.6 5.0 4.4 4.9 5.4 4.8 4.8 4.3 5.8 5.7 5.4 5.1
##  [19] 5.7 5.1 5.4 5.1 4.6 5.1 4.8 5.0 5.0 5.2 5.2 4.7 4.8 5.4 5.2 5.5 4.9 5.0
##  [37] 5.5 4.9 4.4 5.1 5.0 4.5 4.4 5.0 5.1 4.8 5.1 4.6 5.3 5.0 7.0 6.4 6.9 5.5
##  [55] 6.5 5.7 6.3 4.9 6.6 5.2 5.0 5.9 6.0 6.1 5.6 6.7 5.6 5.8 6.2 5.6 5.9 6.1
##  [73] 6.3 6.1 6.4 6.6 6.8 6.7 6.0 5.7 5.5 5.5 5.8 6.0 5.4 6.0 6.7 6.3 5.6 5.5
##  [91] 5.5 6.1 5.8 5.0 5.6 5.7 5.7 6.2 5.1 5.7 6.3 5.8 7.1 6.3 6.5 7.6 4.9 7.3
## [109] 6.7 7.2 6.5 6.4 6.8 5.7 5.8 6.4 6.5 7.7 7.7 6.0 6.9 5.6 7.7 6.3 6.7 7.2
## [127] 6.2 6.1 6.4 7.2 7.4 7.9 6.4 6.3 6.1 7.7 6.3 6.4 6.0 6.9 6.7 6.9 5.8 6.8
## [145] 6.7 6.7 6.3 6.5 6.2 5.9
\end{verbatim}

La función \texttt{summary()} muestra un resumen de un conjunto de datos

\begin{Shaded}
\begin{Highlighting}[]
\KeywordTok{summary}\NormalTok{(iris)}
\end{Highlighting}
\end{Shaded}

\begin{verbatim}
##   Sepal.Length    Sepal.Width     Petal.Length    Petal.Width   
##  Min.   :4.300   Min.   :2.000   Min.   :1.000   Min.   :0.100  
##  1st Qu.:5.100   1st Qu.:2.800   1st Qu.:1.600   1st Qu.:0.300  
##  Median :5.800   Median :3.000   Median :4.350   Median :1.300  
##  Mean   :5.843   Mean   :3.057   Mean   :3.758   Mean   :1.199  
##  3rd Qu.:6.400   3rd Qu.:3.300   3rd Qu.:5.100   3rd Qu.:1.800  
##  Max.   :7.900   Max.   :4.400   Max.   :6.900   Max.   :2.500  
##        Species  
##  setosa    :50  
##  versicolor:50  
##  virginica :50  
##                 
##                 
## 
\end{verbatim}

\begin{Shaded}
\begin{Highlighting}[]
\KeywordTok{summary}\NormalTok{(iris}\OperatorTok{$}\NormalTok{Sepal.Width)}
\end{Highlighting}
\end{Shaded}

\begin{verbatim}
##    Min. 1st Qu.  Median    Mean 3rd Qu.    Max. 
##   2.000   2.800   3.000   3.057   3.300   4.400
\end{verbatim}

\hypertarget{ejercicios}{%
\subsection{1.1 Ejercicios}\label{ejercicios}}

Busca en la ayuda como configurar el número de cifras significativas que
muestra la función \texttt{summary()} y cámbialo para que sólo muestre
un decimal:

\begin{Shaded}
\begin{Highlighting}[]
\CommentTok{#summary(iris ...)#Modifica esta línea}
\end{Highlighting}
\end{Shaded}

\hypertarget{cargando-un-paquete-en-el-entorno}{%
\subsection{1.2 Cargando un paquete en el
entorno}\label{cargando-un-paquete-en-el-entorno}}

La función ´bkde()´ calcula una estimación de la densidad de la
distribución de un conjunto de datos. Es una función de un paquete que
no está cargado por defecto. Si la intentamos invocar no va a funcionar:

\begin{Shaded}
\begin{Highlighting}[]
\CommentTok{#estimacion_densidad = bkde(iris$Sepal.Width)}
\end{Highlighting}
\end{Shaded}

Podemos buscar en Que paquete esta con el comando \texttt{??}

\begin{Shaded}
\begin{Highlighting}[]
\NormalTok{??bkde}
\end{Highlighting}
\end{Shaded}

En la ayuda podemos ver que está en el paquete \texttt{KernSmooth}. Para
cargar un paquete empleamos la función \texttt{library()} y el nombre
del paquete, que puede ir entre comillas o sin comillas:

\begin{Shaded}
\begin{Highlighting}[]
\KeywordTok{library}\NormalTok{(KernSmooth)}
\end{Highlighting}
\end{Shaded}

\begin{verbatim}
## KernSmooth 2.23 loaded
## Copyright M. P. Wand 1997-2009
\end{verbatim}

\begin{Shaded}
\begin{Highlighting}[]
\NormalTok{estimacion_densidad =}\StringTok{ }\KeywordTok{bkde}\NormalTok{(iris}\OperatorTok{$}\NormalTok{Sepal.Width)}
\KeywordTok{plot}\NormalTok{(estimacion_densidad)}
\end{Highlighting}
\end{Shaded}

\includegraphics{1.4Paquetes_files/figure-latex/unnamed-chunk-9-1.pdf}

\hypertarget{instalando-un-paquete}{%
\subsection{1.3 Instalando un paquete}\label{instalando-un-paquete}}

El paquete \texttt{dplyr} contiene funciones útiles para manipular
datos. Pero no está instalado por defecto, sino que hay que instalarlo
(Para ejecutar el comando tendrás que quitar el comentario. Pero no
podrás compilar todo el archivo .Rmd si no está comentado. La
instalación de paquetes es algo que sólo es necesario realizar una vez,
y no es algo que se deba hacer en Markdow):

\begin{Shaded}
\begin{Highlighting}[]
\CommentTok{#install.packages("dplyr")}
\end{Highlighting}
\end{Shaded}

Una vez instalado, hay que cargarlo:

\begin{Shaded}
\begin{Highlighting}[]
\KeywordTok{library}\NormalTok{(dplyr)}
\end{Highlighting}
\end{Shaded}

\begin{verbatim}
## Registered S3 methods overwritten by 'tibble':
##   method     from  
##   format.tbl pillar
##   print.tbl  pillar
\end{verbatim}

\begin{verbatim}
## 
## Attaching package: 'dplyr'
\end{verbatim}

\begin{verbatim}
## The following objects are masked from 'package:stats':
## 
##     filter, lag
\end{verbatim}

\begin{verbatim}
## The following objects are masked from 'package:base':
## 
##     intersect, setdiff, setequal, union
\end{verbatim}

Ahora podemos usar sus funciones. Seleccionamos solo las flores
Virginica con la función \texttt{filter()}

\begin{Shaded}
\begin{Highlighting}[]
\KeywordTok{filter}\NormalTok{(iris,Species}\OperatorTok{==}\StringTok{"virginica"}\NormalTok{)}
\end{Highlighting}
\end{Shaded}

\begin{verbatim}
##    Sepal.Length Sepal.Width Petal.Length Petal.Width   Species
## 1           6.3         3.3          6.0         2.5 virginica
## 2           5.8         2.7          5.1         1.9 virginica
## 3           7.1         3.0          5.9         2.1 virginica
## 4           6.3         2.9          5.6         1.8 virginica
## 5           6.5         3.0          5.8         2.2 virginica
## 6           7.6         3.0          6.6         2.1 virginica
## 7           4.9         2.5          4.5         1.7 virginica
## 8           7.3         2.9          6.3         1.8 virginica
## 9           6.7         2.5          5.8         1.8 virginica
## 10          7.2         3.6          6.1         2.5 virginica
## 11          6.5         3.2          5.1         2.0 virginica
## 12          6.4         2.7          5.3         1.9 virginica
## 13          6.8         3.0          5.5         2.1 virginica
## 14          5.7         2.5          5.0         2.0 virginica
## 15          5.8         2.8          5.1         2.4 virginica
## 16          6.4         3.2          5.3         2.3 virginica
## 17          6.5         3.0          5.5         1.8 virginica
## 18          7.7         3.8          6.7         2.2 virginica
## 19          7.7         2.6          6.9         2.3 virginica
## 20          6.0         2.2          5.0         1.5 virginica
## 21          6.9         3.2          5.7         2.3 virginica
## 22          5.6         2.8          4.9         2.0 virginica
## 23          7.7         2.8          6.7         2.0 virginica
## 24          6.3         2.7          4.9         1.8 virginica
## 25          6.7         3.3          5.7         2.1 virginica
## 26          7.2         3.2          6.0         1.8 virginica
## 27          6.2         2.8          4.8         1.8 virginica
## 28          6.1         3.0          4.9         1.8 virginica
## 29          6.4         2.8          5.6         2.1 virginica
## 30          7.2         3.0          5.8         1.6 virginica
## 31          7.4         2.8          6.1         1.9 virginica
## 32          7.9         3.8          6.4         2.0 virginica
## 33          6.4         2.8          5.6         2.2 virginica
## 34          6.3         2.8          5.1         1.5 virginica
## 35          6.1         2.6          5.6         1.4 virginica
## 36          7.7         3.0          6.1         2.3 virginica
## 37          6.3         3.4          5.6         2.4 virginica
## 38          6.4         3.1          5.5         1.8 virginica
## 39          6.0         3.0          4.8         1.8 virginica
## 40          6.9         3.1          5.4         2.1 virginica
## 41          6.7         3.1          5.6         2.4 virginica
## 42          6.9         3.1          5.1         2.3 virginica
## 43          5.8         2.7          5.1         1.9 virginica
## 44          6.8         3.2          5.9         2.3 virginica
## 45          6.7         3.3          5.7         2.5 virginica
## 46          6.7         3.0          5.2         2.3 virginica
## 47          6.3         2.5          5.0         1.9 virginica
## 48          6.5         3.0          5.2         2.0 virginica
## 49          6.2         3.4          5.4         2.3 virginica
## 50          5.9         3.0          5.1         1.8 virginica
\end{verbatim}

Seleccionamos la misma especie, pero solo aquellas que miden entre 2.7 y
6 cm

\begin{Shaded}
\begin{Highlighting}[]
\KeywordTok{filter}\NormalTok{(iris,Species}\OperatorTok{==}\StringTok{"virginica"} \OperatorTok{&}\StringTok{ }\NormalTok{Sepal.Length}\OperatorTok{<}\DecValTok{6} \OperatorTok{&}\StringTok{ }\NormalTok{Sepal.Width}\OperatorTok{<=}\FloatTok{2.7}\NormalTok{)}
\end{Highlighting}
\end{Shaded}

\begin{verbatim}
##   Sepal.Length Sepal.Width Petal.Length Petal.Width   Species
## 1          5.8         2.7          5.1         1.9 virginica
## 2          4.9         2.5          4.5         1.7 virginica
## 3          5.7         2.5          5.0         2.0 virginica
## 4          5.8         2.7          5.1         1.9 virginica
\end{verbatim}

Podemos desinstalar paquetes con el comando remove.packages(). Ojo que
está comentado; puedes probar a descomentarlo y ejecutarlo.

\begin{Shaded}
\begin{Highlighting}[]
\CommentTok{#remove.packages("dplyr")}
\end{Highlighting}
\end{Shaded}

\end{document}
